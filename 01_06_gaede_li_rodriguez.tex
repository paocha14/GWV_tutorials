\documentclass[ngerman]{article}
\usepackage[utf8]{inputenc}
\usepackage[T1]{fontenc}
\usepackage[ngerman]{babel}
\usepackage{amsmath}
\usepackage{amsfonts} 
\usepackage{booktabs}
\usepackage{amssymb}
\usepackage{amsthm}
\usepackage{stmaryrd}
\usepackage{mathtools}
\usepackage{algpseudocode}
\usepackage{algorithm}
\usepackage[a4paper, portrait, left=2.5cm, right=2cm, top=2cm, bottom=3cm]{geometry}
\usepackage{fancyhdr}
\usepackage{calc}
\setlength{\headheight}{15.2pt}
\pagestyle{fancy}
\setlength{\arraycolsep}{2pt}
\lhead{GWV | Tutorial 1 | Mi 12-14 G021-022}
\rhead{Connor Gaede, Yang Li, Jose Rodriguez}
\cfoot{\thepage}
\date{}
\newtheorem*{theorem}{Behauptung}
\newcommand{\N}{\mathbb{N}}
\newcommand{\Z}{\mathbb{Z}}
\newcommand{\R}{\mathbb{R}}
\newcommand{\C}{\mathbb{C}}
\newcommand{\oO}{\mathcal{O}}
\title{GWV Tutorial 1}
\begin{document}
\section*{Exercise 1.1: Search space properties}
\begin{itemize}
\item The first distinction might be important because
\end{itemize}
\section*{Exercise 1.2: Search Space 1}
\subsection*{a)}
\begin{enumerate}
\item First we have to define a model of the state of the two jugs. We suggest
\[
(x_1,x_2)
\]
as a model where $x_i$ is the amount of water in litres in jug $i$, $i\in\lbrace1,2\rbrace$.
\item 
\end{enumerate}
\end{document}
